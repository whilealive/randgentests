% ==================================================================
% FILE     AussagetestDebug.tex
% INFO     -
%
% DATE     17.06.2022
% OWNER    Bischofberger
% ==================================================================

\documentclass{AussagetestClass}

% Ordnerstruktur: Fragesammlung/01-Algebra/ ...
\newcommand*{\Sammelordner}{Fragesammlung}

\newcommand{\Themenliste}{%
  01-Algebra,
  02-Potenzrechnen,
  %03-Lineare-Gleichungen,
  %04-Lineare-Gleichungssysteme,
  %05-Quadratische-Gleichungen,
  %06-Nichtlineare-Gleichungen,
  %07-Lineare-Funktionen,
  %08-Quadratische-Funktionen,
  %09-Potenzfunktionen,
  %10-Polynomfunktionen,
  %11-Exponentialfunktionen,
  %12-Datenanalyse,
  %13-Wahrscheinlichkeitsrechnen,
  %14-Planimetrie-I,
  %15-Planimetrie-II,
  %16-Trigonometrie-I,
  17-Trigonometrie-II,
  %18-Stereometrie,
  %19-Vektorgeometrie,
}%

\newcommand*{\AnzahlWahreAussagen}{3}
\newcommand*{\AnzahlFalscheAussagen}{5}


% ==================================================================
\begin{document}

\directlua{
  require('statementCollator.lua')

  % -----------------------------------------------------------------
  % Dieser Teil erstellt die zufallsgenerierten Tests

  local sep = getPathSeparator()
  local dirlist = csvsplit('\Themenliste')
  addprepoststring(dirlist, lfs.currentdir() .. sep .. '\Sammelordner' .. sep, sep)

  local mixedstatements = collateStatements(dirlist, \AnzahlWahreAussagen, \AnzahlFalscheAussagen)

  printStatements(mixedstatements)
  printSolutions(mixedstatements, true)

  % -----------------------------------------------------------------
  % Dieser Teil erstellt eine Zusammenfassung aller vorhandenen Aussagen

  printAll(lfs.currentdir() .. sep .. '\Sammelordner', true)

  % -----------------------------------------------------------------
}

%\tikz \foreach \fd in \Themenliste \draw (0,0) node {$\fd$};

\end{document}
% ==================================================================
