% ==================================================================
% FILE     Aussagetest.tex
% INFO     -
%
% DATE     01.07.2022
% OWNER    Bischofberger
% ==================================================================

\documentclass{AussagetestClass}

% ------------------
% Optionen:
% ------------------
\newcommand*{\Sammelordner}{Fragesammlung}

\newcommand*{\Oberthema}{01-Algebra}
%\newcommand*{\Oberthema}{02-Potenzrechnen}
%\newcommand*{\Oberthema}{03-Lineare-Gleichungen}
%\newcommand*{\Oberthema}{04-Lineare-Gleichungssysteme}
%\newcommand*{\Oberthema}{05-Quadratische-Gleichungen}
%\newcommand*{\Oberthema}{06-Nichtlineare-Gleichungen}
%\newcommand*{\Oberthema}{07-Lineare-Funktionen}
%\newcommand*{\Oberthema}{08-Quadratische-Funktionen}
%\newcommand*{\Oberthema}{09-Potenzfunktionen}
%\newcommand*{\Oberthema}{10-Polynomfunktionen}
%\newcommand*{\Oberthema}{11-Exponentialfunktionen}
%\newcommand*{\Oberthema}{12-Datenanalyse}
%\newcommand*{\Oberthema}{13-Wahrscheinlichkeitsrechnen}
%\newcommand*{\Oberthema}{14-Planimetrie-I}    % Dreiecke/Vierecke
%\newcommand*{\Oberthema}{15-Planimetrie-II}   % Kreis, Bogenmass, Ähnlichkeit
%\newcommand*{\Oberthema}{16-Trigonometrie-I}  % am rechtwinkligen Dreieck
%\newcommand*{\Oberthema}{17-Trigonometrie-II} % am allgemeinen Dreieck, Einheitskreis, Graphen
%\newcommand*{\Oberthema}{18-Stereometrie}
%\newcommand*{\Oberthema}{19-Vektorgeometrie}

\newcommand*{\AnzahlWahreAussagen}{3}
\newcommand*{\AnzahlFalscheAussagen}{5}


% ==================================================================
\begin{document}

%\section*{Aussagetest, \Oberthema}
\section*{Aussagetest, Algebra}
Kreuzen Sie alle wahren Aussagen an.

\vskip\baselineskip
%\clearpage

\directlua{
  require('statementCollator.lua')
  local sep = getPathSeparator()

  % -----------------------------------------------------------------
  % Dieser Teil erstellt die zufallsgenerierten Tests

  local mixedstatements = collateStatements(lfs.currentdir() .. sep .. '\Sammelordner' .. sep .. '\Oberthema' .. sep , \AnzahlWahreAussagen, \AnzahlFalscheAussagen)

  printStatements(mixedstatements)
  printSolutions(mixedstatements, false)

  % -----------------------------------------------------------------
  % Dieser Teil erstellt eine Zusammenfassung aller vorhandenen Aussagen

  printAll(lfs.currentdir() .. sep .. '\Sammelordner', true)

  % -----------------------------------------------------------------
}

\end{document}
% ==================================================================

